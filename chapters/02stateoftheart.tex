\section{L'évolution de la coopération}

\subsection{L'évolution}
\label{ssec:evolution}

La coopération est un problème important en biologie évolutionnaire, qui étudie l'apparition et le maintien de traits morphologiques et comportementaux dans le vivant. D'après la théorie de la biologie évolutionnaire, chaque individu possède un certain nombres de caractéristiques qui peuvent être soit avantageuses soit désavantageuses pour l'individu dans son environnement comparé à ses pairs. Une caractéristique est dite avantageuse si elle permet à l'individu d'être plus adapté à son environnement que ses pairs, c'est-à-dire avoir une meilleure fitness. Cela signifie que l'individu a plus de capacité à survivre et à se reproduire s'il possède cette caractéristique comparé à ses pairs.

Si cette caractéristique est transmissible par reproduction, puisque l'individu se reproduit plus que les autres membres de la population, la caractéristique plus adaptée est de plus en plus fréquente dans la population. La descendance de cet individu est elle aussi plus adaptée que la descendance des autres individus, elle a donc un plus grand succès reproducteur. La descendance de la descendance de l'individu adapté est elle aussi plus adaptée comparé à celle de la descendance de la descendance des autres individus et ainsi de suite. La fréquence de la nouvelle caractéristique dans la population augmente au fil des générations au point que la caractéristique soit virtuellement dans toute la population. La caractéristique s'est fixée dans la population. C'est le mécanisme de sélection naturelle.

Ainsi, par sélection naturelle et reproduction, les caractéristiques les plus adaptées se propagent et se fixent dans les populations. Nous avons pris ici le cas simple d'une unique caractéristique, dans un environnement stationnaire. La fixation de caractéristiques peut être beaucoup plus complexe, parce que les caractéristiques qui définissent un individu peuvent avoir des interactions entre elles, ou bien l'environnement dans lequel évolue la population peut changer, soit de manière exogène au processus d'adaptation des individus, ou bien parce que le comportement des individus suite à la fixation de nouvelles caractéristiques viennent modifier celui-ci.

Nous avons ici présenté le premier ingrédient des mécanismes de l'évolution, la sélection. Le second élément est la variation. Nous avons précédemment postulé qu'un individu avait une caractéristique différente des autres membres de sa population. Comment peut-on avoir des variations de caractéristiques ? Précédemment, nous avons fait l'hypothèse implicite d'une transmission parfaite des caractéristiques du parent à la descendance. Faisons maintenant l'hypothèse explicite contraire : Il est possible que lors de la reproduction, il y ait une probabilité que la caractéristique transmise soit légèrement altérée. Cette altération peut avoir des conséquences marginales (la caractéristique est plus ou moins fortement exprimée), ou elle peut même avoir des conséquences importantes, comme changer complètement la nature de la caractéristique. Cette nouvelle version de la caractéristique peut être elle aussi adaptée ou non, et donc subir le même processus de sélection que décrit précédemment. 

Ainsi, le mécanisme de variation introduit de nouvelles caractéristiques dans la population, tandis que le processus de sélection vient conserver les caractéristiques les plus adaptées qui vont se fixer dans la population au détriment des moins adaptées qui vont elles s'éteindre. Ce processus, l'évolution, conduit un processus d'optimisation. À chaque génération, les caractéristiques qui maximisent la survie et la reproduction des individus se propagent.


\begin{verbatim}
ajouter exemples?
\end{verbatim}

\subsection{Le problème de la coopération}

En biologie de l'évolution, la coopération peut être définie comme le fait d'agir pour apporter un bénéfice à un autre individu. C'est un comportement très déroutant pour la biologie évolutionnaire. En effet, comment un tel comportement peut-il être adapté ? Un individu qui coopère dépense son temps et son énergie afin d'aider un autre individu. Il dépense des ressources pour augmenter la fitness, le succès reproducteur d'un autre. S'il est aisé de comprendre qu'être le bénéficiaire d'une coopération soit adapté et que ce comportement puisse être conservé par sélection naturelle, si tant est que ce serait transmissible, il est déroutant que la caractéristique \emph{d'être} coopérateur puisse être adapté. Il est difficile d'imaginer que cette caractéristique puisse être conservée par sélection naturelle. 

Pourtant, de nombreux cas de coopération sont observés dans le vivant. Tout d'abord, l'un des exemples les plus marquants pourrait être les insectes eusociaux, dont font partie les fourmis ou les abeilles. Pour ces espèces, les membres de colonies entières se coordonnent et travaillent ensemble afin de permettre le succès reproducteur de leurs reines. C'est un cas de coopération que nous développerons dans la section~\ref{ssec:altruism_kin}~\emph{\nameref{ssec:altruism_kin}}. Des comportements coopératifs sont aussi observés chez les gobemouches noirs, qui aident les autres gobemouches lorsqu'ils se font chassés par une chouette en attaquant collectivement le prédateur.  Ainsi, les gobemouches qui viennent aider la proie risquent leurs vies pour sauver celle d'un autre individu. Enfin, les humains sont de très grands coopérateurs. Ils possèdent des structures sociales complexes, produisent leurs ressources ensemble, les échangent, donnent à des organisations caritatives.


C'est comportements coopératifs sont comme dit précédemment très déroutant. Par exemple, il est aisé d'imaginer qu'un gobemouche noir qui n'attaque pas le prédateur d'un de ses partenaires pourrait avoir une meilleure fitness en ne mettant pas en danger sa vie. Comment expliquer que ces comportements n'ait pas été contre-sélectionnés ? Cette question est très fortement étudiée en biologie de l'évolution. De nombreux travaux ont identifié différents mécanismes qui permettrait de rendre les comportements de coopération viables. Ces mécanismes sont détaillés dans les prochaines sections.



\subsection{L'altruisme et l'apparentement}
\label{ssec:altruism_kin}

Une première explication des comportements coopératifs est la sélection de parentèle \citep{Hamilton1964}. Les explications de l'évolution donnés dans la section~\ref{ssec:evolution}~\emph{\nameref{ssec:evolution}} étaient centrées sur l'individu. Cependant, la propagation de caractéristiques est plus complexe que cela. L'individu n'est qu'un véhicule transportant la caractéristique, et celle-ci . Le support de cette caractéristique dans le vivant est le gène. Le gène code cette caractéristique. Le support même. \todo{compléter} 

Ainsi, il peut être rentable pour un individu d'aider un de ses apparentés, si le bénéfices $b$ pour le bénéficiaires pondérées par le degré d'apparentement $r$ surpasse les coûts engagés par l'acteur $c$, alors le comportement altruistique est stable. C'est la règle d'Hamilton définie dans l'équation~\ref{eq:hamiltonrule}.

\begin{equation}
r \times b > c \label{eq:hamiltonrule}
\end{equation}


Dans ce cas là, c'est aider des parties de soi plutôt que d'aider qqun d'autre. Cette implémentation de la coopération est mise en places chez les insectes eusociaux tel que les fourmis ou les abeilles, où le degré d'apparentement entre les individus d'une même colonie est de 75\%, ou bien entre les cellules d'organismes multicellulaires, où le niveau d'apparentement est ici de 100\%.

Cependant, cela permet de comprendre qu'un sous-ensemble des comportements observés dans le vivant. Comment les comportements de coopération entre individus d'espèces différentes ont pu évoluer ? De même, comment les comportements entre individus d'une même espèce mais non-apparentés, comme on l'observe chez l'Humain ou chez les chauves-souris vampires par exemple peuvent se développer.

\subsection{La réciprocité}

Un gène ne peut être conservée au cours de l'évolution uniquement si ce gène déploie un phénotype plus adapté que le phénotype de ses concurrents pour sa reproduction. Comme vu dans la section~\ref{ssec:altruism_kin}~\emph{\nameref{ssec:altruism_kin}}, cela peut se faire de manière indirecte, ce qui permet d'expliquer les comportements de coopération entre apparentés. Dans le cas de coopération entre non-apparentés, c'est donc que le comportement de coopération exprimé par le gène d'un acteur apporte bien un bénéfice à cet acteur lui-même. En effet, sans sélection de parentèle, l'expression du gène ne possède aucun moyen de déterminer si le bénéficiaire de la coopération possède lui aussi le même exemplaire de ce gène. Quand bien même le bénéficiaire posséderait les mêmes expressions phénotypiques que l'acteur, cela ne garantit pas la propagation du gène responsable pour ce comportement. \todo{green beard \cite{Dawkins1976}} Ainsi, un gène qui code pour un comportement coopératif, c'est-à-dire qui code un comportement aillant un coût pour l'acteur et un bénéfice pour le bénéficiaire, doit forcément apporter un bénéfice comparé à ses concurrents.

C'est le cas si le comportement coopératif de l'acteur entraîne un changement de comportement sur le bénéficiaire, comme le propose \citet{Trivers1971}. Ainsi, il est intéressant d'agir de manière coopérative avec un individu qui le sera lui aussi avec nous en retour. C'est la coopération conditionnelle ou la réciprocité. Cette réciprocité peut être soit positive, c'est à dire que le bénéficiaire coopère avec l'acteur en réponse à la coopération ; soit négative, le bénéficiaire punit l'acteur si celui-ci ne coopère pas. Dans les deux cas, il est dans l'intérêt de l'acteur de coopérer, puisque cela maximise son gain en vu de la réaction du bénéficiaire. Notons cependant qu'agir de manière coopérative n'a de sens ici uniquement si cela a un effet sur le comportement du bénéficiaire. Si le bénéficiaire aurait de toute façon aider l'acteur après coût, ou bien si le bénéficiaire lui aurait de toute façon causer dommage, alors l'acteur n'a aucun intérêt à coopérer en faveur du bénéficiaire. C'est en cela qu'il y a réciprocité. Les comportements de réciprocité peuvent être implémenté très facilement et sont particulièrement robuste, comme l'a montré \citet{Axelrod1981}. Ainsi, dans le cadre d'une situation de coopération modélisable comme un dilemme du prisonnier, avec interactions répétées, le comportement de tit-for-tat est une stratégie de réciprocité extrêmement robuste. De même, les comportements de punition chez les poissons nettoyeurs avec leurs clients \citep{bsharyxx}. Les comportements tit-for-tat se retrouvent aussi avec les gobemouches noirs \citep{Krams2008}, qui ne viennent défendre que les gobemouches noirs qui les ont déjà défendu auparavant, mais jamais ceux qui ne sont jamais venu les aider alors qu'ils en avaient besoin.

Il y a ainsi deux types de réciprocité directe: Le Partner Fidelity Feedback et le Partner Choice \citep{Sachs2004}.

\subsection{Le choix du partenaire et les marchés biologiques}

\xx{Parler d'opportunités de coopération !!!}

Parmi les implémentations de la coopération par réciprocité, le choix du partenaire semble être apparu de nombreuses fois et être un mécanisme particulièrement efficace. Dans l'espèce Humaine, il joue un rôle prépondérant dans le maintien des comportements de coopération. Le choix du partenaire permet l'apparition et le maintient de la coopération. Pour le comprendre, ne nous centrons plus sur l'acteur de la coopération, mais le bénéficiaire. Dans une tâche collective, il est toujours pertinent pour le bénéficiaire de la tâche collective de se mettre avec le meilleur acteur possible, c'est-à-dire l'acteur qui lui permettra d'obtenir le plus gros gain. Puisque pour réaliser la tâche, qui est \emph{collective}, l'acteur a aussi besoin du bénéficiaire, il est alors dans l'intérêt des acteurs d'être le plus coopératif possible afin d'être choisi par un bénéficiaire. Ainsi, il y a une pression a être coopératif afin d'être choisi par un bénéficiaire. Cette pression est d'autant plus forte si le nombre d'acteurs est particulièrement grand par rapport au nombre de bénéficiaire. Il y a un effet de marché \citep{Noe1994}.

Imaginons maintenant une population d'individus qui cherche à être avec le meilleur partenaire possible, et que tous les individus présents dans cette population sont égoïstes. Dans cette population apparaît un mutant qui coopère plus que les autres. Ce mutant va être particulièrement recherché par les autres individus de la population. Il va donc interagir beaucoup et obtenir de nombreux gains. De la même manière, puisque de nombreux individus vont vouloir coopérer avec lui, il va pouvoir être sélectif. Il va pouvoir refuser les interactions avec les moins efficaces pour choisir les interactions avec les plus performants. Il y a donc un assortative matching qui se met en place. Les individus les plus performants qui interagissent ensemble vont recevoir de très nombreux bénéfices de leurs interactions. Ces bénéfices ont un impact positif sur leur fitness et ils vont se propager dans la population. Le fait d'être coopératif va se fixer dans la population et tous les individus seront des coopérateurs.

Le choix du partenaire

Ces mécanismes de choix du partenaire se retrouvent dans la nature. C'est ainsi eux qu'on observe dans les mutualismes inter-spécifique chez les cleaners fishes et leurs clients \citep{Bshary2002b} \todo{détailler} ou bien chez les mutualismes legumes-rhizobium \todo{source + détail}.


partner switching

\cite{Aktipis2011}


\begin{verbatim}
Choisir un partenaire pour le gain qu'il nous apporte, non pas pour l'aider lui. crée une boucle
- grooming
- vampire bats
- cleaner/client
\end{verbatim}

\subsection{Pourquoi la coopération n'est pas partout ?}

Bien que nous nous demandions en début de ce chapitre comment la coopération pourrait évoluer, après l'étude des différents mécanismes qui puissent l'implémenter, c'est maintenant la question contraire qui apparaît. Comment se fait-il que la coopération entre non-apparentés soit en fait si rare dans la nature ? En effet \xx{exemple de pas coopération même si c'est tendu}, et pourtant le choix du partenaire est un mécanisme particulièrement puissant chez l'Homme \todo{cite}. 

Quels sont les paramètres qui pourrait empêcher l'apparition de réciprocité ?

Tout d'abord, il faut souligner le problème du bootstrapping. S'il est aisé de comprendre comment les mécanismes de réciprocité peuvent maintenir les comportements de coopération, il est en fait plus compliqué d'expliquer comment ce mécanisme peut apparaître de lui-même. En effet, la réciprocité -- tant la Partner Fidelity Feedback et le Partner Choice -- requiert.

bootstrapping
Dans ce cas, pourquoi la coop n'est pas partout? Qu'est-ce qui la bloque ?
Objectif de la thèse : capturer le pourquoi la coop c'est difficile


%%%%%%%%%%%%%%%
%%%%%%%%%%%%%%%

\section{Modèles et méthodes}

\subsection{Les modèles analytiques}

Théorie des jeux évolutionnaire

\subsection{Les modèles agents centrés}

McNamara, Aktipis(?)

\subsection{La robotique évolutionniste}

importance des problèmes de coordination et mécanistes.
Intéressant d'avoir des robots capables de coop

\subsection{Le mapping génotypique-phénotypique}

%%%%%%%%%%%%%%
%%%%%%%%%%%%%%

\section{Objectif de la thèse}

\begin{verbatim}
    
comprendre pourquoi la coopération bloque, pourquoi c'est difficile à obtenir
- Utilisation de modèles agents centrés et robotiques pour capturer les difficultés présentes et que ne capturent pas les autres modèles
-> Contraintes de coordinations, d'accès au ressources, de densité, de navigation
\end{verbatim}

